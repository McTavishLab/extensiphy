\documentclass[a4paper,10pt]{article}
\usepackage[utf8]{inputenc}
\usepackage{hyperref}

%opening
\title{Genetree updating plan}
\author{}

\begin{document}

\maketitle

\begin{itemize}
 \item [\textbf{End goal:}]
Trees continually updated as new seqs are posted to database
 \item[\textbf{Concept:}]
Some set of (10? 100?) gene trees and alignments that are continually updated by scraping genbank or SRA.\\
Treat trees as large but taxon limited to attempt to minimize alignment/orthology issues.\\
Any alignable sequence posted to database from a that taxonomic group is added to the aligment and a new tree estimated.
 \item[\textbf{Case studies:}]
16s RNA in Cyanobactria, building on \href{https://tree.opentreeoflife.org/curator/study/view/pg_2739}{Schirrmeister 2011 pg\_2739}\\
ITS and LSU in Ascomycota, building on \href{https://tree.opentreeoflife.org/curator/study/view/pg_873}{Crous 2012 pg\_873}\\
 \item[\textbf{Deadlines:}]
Have basic loop (NCBI scraping, alignment, placement, tree estimation) working for these two trees by August OT Software meeting.\\
Following that meeting discuss with organismal folks priorities for genes, trees, taxa that are well suited to this approach.\\
 \item[\textbf{Extensions/in parallel:}]
Improve tree updating algorithms to reduce lag time due to tree estimation step.\\
 \item[\textbf{Questions:}]
Add sequences only if represent new tips not already represented in this or any other trees?\\
How to select trees/loci for updating?\\
How broad/narrow should trees for updating be?\\
Tests for error/prarology, seq misnaming? (e.g. Alexey and JiaJie's work)\\
\end{itemize}

\end{document}
