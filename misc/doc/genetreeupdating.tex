\documentclass[a4paper,10pt]{article}
\usepackage[utf8]{inputenc}
\usepackage{hyperref}

%opening
\title{Genetree updating plan}
\author{}

\begin{document}

\maketitle

\begin{itemize}
 \item [\textbf{End goal:}]
Trees continually updated as new seqs are posted to database
 \item[\textbf{Concept:}]
Some set of (10? 100?) gene trees and alignments that are continually updated by scraping genbank or SRA.\\
Treat trees as large but taxon limited to attempt to minimize alignment/orthology issues.\\
Any alignable sequence posted to database from a that taxonomic group is added to the aligment and a new tree estimated.
 \item[\textbf{Case studies:}]
16s RNA in Cyanobactria, building on \href{https://tree.opentreeoflife.org/curator/study/view/pg_2739}{Schirrmeister 2011 pg\_2739}\\
ITS and LSU in Ascomycota, building on \href{https://tree.opentreeoflife.org/curator/study/view/pg_873}{Crous 2012 pg\_873}\\
 \item[\textbf{Deadlines:}]
Have basic loop (NCBI scraping, alignment, placement, tree estimation) working for these two trees by August OT Software meeting.\\
Following that meeting discuss with organismal folks priorities for genes, trees, taxa that are well suited to this approach.\\
 \item[\textbf{Extensions/in parallel:}]
Improve tree updating algorithms to reduce lag time due to tree estimation step.\\
 \item[\textbf{Questions:}]
Add sequences only if represent new tips not already represented in this or any other trees?\\
Use BLAST or annotations to choose seqs?\\
How to select trees/loci for updating?\\
How broad/narrow should trees for updating be?\\
Tests for error/prarology, seq misnaming? (e.g. Alexey and JiaJie's work)\\
\end{itemize}


Plan:

inputs: tree, alignment, target taxon list (ott <-> ncbi)
Target taxon list not exceed ?? value?



Step 1: 

get listt of wanted ott_ids from tree.

Get tree with ott_ids

https://tree.opentreeoflife.org/curator/study/view/pg_2739/?tab=trees&tree=tree6601
curl http://api.opentreeoflife.org/v2/study/pg_2739.tre/?tip_label=ot:ottid > cyan.tre

curl http://api.opentreeoflife.org/v2/study/pg_2739/otus

curl http://api.opentreeoflife.org/v2/study/pg_2739/otus > cyan.otus


curl -X POST http://api.opentreeoflife.org/v2/tree_of_life/mrca -H "content-type:application/json" -d '{"ott_ids":[412129, 536234]}'

from that mrca, get all tip ID's
curl -X POST http://api.opentreeoflife.org/v2/tree_of_life/subtree -H "content-type:application/json" -d '{"ott_id":3599390}'
withe th ott id as teh return from the previous func.

Concept: fill in existing trees, rather than adding seq from outside? SImplify alignment issues, homology etc.

pull OTT id's off that subtree. Inlcude only OTT_ID's not already represented by THIS phylogeny.


BLAST (every?) seq from the alignemnt and pull down genbank records. Q. Is local DB needed?

if OTT_ID on most wanted list, add to alignement, proceed.

Proceed with alignment, placement and tree inference as set up in phycorder.


_taxon_labels 


excluding re-used taxa? not to be added? 

Save them for another time...
for testing / placement 


Some set of trees and some 

RBCL ... mt genome for vert groups...





\end{document}
