\documentclass[a4paper,10pt]{article}
\usepackage[utf8]{inputenc}

%opening
\title{Streaming phylogenies from whole genome sequence data}
\author{EJM, MTH, AS}

\begin{document}

\maketitle

\begin{abstract}

\end{abstract}

While whole genome sequencing is becoming affordable, and many novel species are being sequenced, 
these data are not straightforward to add into pre-existing estimates o phylogenetic relationships.

Much time and effort has been devoted to developing single copy nuclear genes, sequencing them, 
and using them to build phylogenies in order to understand species relationships.
This tool simplifies the process required to go from sequencing an individual and placing it in it's appropriate phylogenetic context.
In contrast to barcoding or blast based methods, this is not an allele-matching approach.
Rather it is a modified assembly approach, which assembles loci homologous to those already existing in phylogenies.

By using pre-existing phylogenies an alignments this approach leverages previous work dedicated to selecting
markers with the appropriate levels of variation to understand relationships among the taxa of interest.
By assembling the full locus using existing loci as a reference, 
full maximum likelihood and phylogenetic methods can be applied to the new alignment.

In contrast to methods that discover homologous loci using properties of the entire data set.
e.g. (SISES, aTRAM) or SNP discovery methods (KSNP ...?) which rely on shared qualities of the entire data set, 
this method captures the necessary information contained in loci homologous to existing alignments

It can be used as a standard 'barcoding' approach if the input tree and alignment are a barcoding gene.

Advantages of this method include: 
rapid filtering of whole genome data sets.
incorporation of new taxa into phylogenies, as opposed to a purely placement based approach
does not require uniform sequencing technology, or sampling plan.
the outputs can easily be shared and combined, as they are full alignments, not just SNP loci.



\subsection{Methods}
This pipeline uses an existing tree and alignment to call homologous loci from a set of NGS reads, and place that taxon in the tree.

It treats each sequence as a potential reference to assemble a homologous locus from the query reads.

\subsubsection{First pass mapping}. (Required) Use bowtie2 to map reads to all loci found in the alignment.
This is a coarse filtering step which a) determines if there is a potentially appropriate reference found in this alignment, 
and b) ranks the potential reference loci buy the number of reads it is possible to map.

\subsubsection{Read alignment and placing} (optional) Align all mapped reads to the reference alignment using PaPaRa (ref) and place those in a phylogenetic tree using EPA.
This step may be used to test if there are reads from multiple biologically relevant clusters.
(problem - conserved seqs will drop to bottom of tree I think, even if only one SPp)

\subsubsection{Consensus locus calling, alignment, and placement} Re-map query reads to best reference genome, using bowtie2 local alignment and call consensus sequence. 


\subsubsection{Alternate consensus locus calling} 
This repeats the consensus locus calling step using an alternate locus as the reference.
By returning both the sequences and the phylogenetic placements which result from the use of an alternate reference, 
it is possible to examine biases due to selection of reference locus.

\subsubsection{Full phylogenetic reanalysis}
If a researcher wishes to continue adding taxa to this tree, a full phylogenetic analysis can be performed using the new alignment.
Following this full ML search, this tree and alignment can be used as the reference data set to add more taxa to the tree.


\subsection{Discussion}
Continually updated phylogenetic trees, with little human input required between steps, 
is necessary for phylogenetics to keep pace with current sequencing.

While whole genome based approaches are becoming available for phylogenetics, many of these rely on dubious inference of homology among groups, 
and are not able to leverage the many years of past phylogenetic effort which has been applied to selecting appropriate markers and understanding relationships in groups of interest.

By building up alignments of homologous 

[issues with SNP based approaches]

[importance of full marker seqs?]

[some blather about phylogenetic models]


\end{document}
